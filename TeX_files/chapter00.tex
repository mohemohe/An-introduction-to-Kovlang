\chapter*{まえがき}
まず,このようなくだらない(と言っては多方面に失礼かもしれないが)本を手にとった皆さんに感謝したい.Twitterのタイムラインで広く使用されている「こふ語」は平易なことから,あまりこふ語について調べようとする人は少ない.その結果,こふ語の変換規則とは異なる,「似非こふ語」が増えてしまっている.そのような状況において,正しいこふ語を学ぼうとする意欲のある人達はかけがえのない存在である.

\vspace{2em}

さて,筆者がこふ語と出会ったのは今から1年と少し前,2013年の8月のことである.何故 @kenkov\footnote{https://twitter.com/kenkov/} と出会ったのかについては思い出せないが,恐らく大したことではないので考えるのをやめる.\par
当時はまだこふ語という概念がなく,現在のような文章としてのこふ語というよりは,むしろ単語の羅列に近いものだった.それから時間を経て,現在のこふ語が生まれた.そのような事実に注目すると,こふ語は既存の日本語を改善し,進化した言語であるといえる.\par
そう考えると,こふ語を理解するためには,基礎的な日本語力が必要である.それはつまり,日本語を話し,読み,書くことができる力である.もし,この本を読みすすめていったときに,なにか理解できないことがあったならば,日本語に立ち返って考えてみてほしい.\par
日本語を理解することが,こふ語の理解への一番の近道なのだから.